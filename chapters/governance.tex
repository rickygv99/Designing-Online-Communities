\documentclass[class=book, crop=false]{standalone}

%% Image paths
\usepackage{graphicx}
\graphicspath{{images/}}

%% Language and font encodings
\usepackage[english]{babel}
\usepackage[utf8x]{inputenc}
\usepackage[T1]{fontenc}

%% Sets page size and margins
\usepackage[a4paper,top=3cm,bottom=2cm,left=3cm,right=3cm,marginparwidth=1.75cm]{geometry}

%% Sets epigraph style
\usepackage{epigraph}
\setlength\epigraphwidth{.8\textwidth}
\setlength\epigraphrule{0pt}

%% Sets line style
\linespread{1.3}

%% Key term command
\usepackage{marginnote}
\providecommand{\keyterm}[1]{\textbf{#1}\marginnote{\scriptsize \textbf{#1}}}

\begin{document}

Moderators\index{moderation} themselves usually don't make the decisions that determine what policies to enforce and what type of content to allow on their platforms. So who does make the decisions? Typically, the line determining what content is permissible has been very blurry. Social systems strive to balance free speech\index{free speech} and a safe environment but tend to be inconsistent in deciding what is allowed and what isn't. What one moderator might feel is acceptable another might feel violates their policies. This inconsistency has led to allegations of bias and discrimination against many tech companies, including Facebook, Twitter and Google.\index{discrimination}

Centivany, A. 2016. "Values, Ethics and Participatory Policymaking in Online Communities." In Proceedings of the Association for Information Science and Technology 53 (1): 1-10.\\
 * Policy plays an important role in translating community values into user interaction.

Fiesler, C. et al. 2016. "Reality and Perception of Copyright Terms of Service for Online Content Creation." In Proceedings of the ACM Conference on Computer Supported Cooperative Work \& Social Computing (CSCW), 1450-1461.\\
 * Terms of Service / privacy policies are typically legally binding yet rarely read and difficult to understand, despite the fact that members of online communities may react badly if they were aware of them.\\
 * Policies can be highly variable across websites.

Kraut, R.E., and Resnick, P. 2014. Building Successful Online Communities: Evidence-Based Social Design. Boston, MA: MIT Press.\\
 * Rules can stop unwanted behavior but they can also deter contribution if community members feel stifled.

Fiesler, C et al. 2015. "Understanding Copyright in Online Creative Communities." In Proceedings of the ACM Conference on Computer Supported Cooperative Work \& Social Computing (CSCW), 116-29.\\
 * Conflicts between multiple sources of rules can result in confusion and disagreement.

\section{Autocracy and technocracy}

Fiesler, Casey et al. 2018. "Reddit rules! characterizing an ecosystem of governance." In Twelfth International AAAI Conference on Web and Social Media.\\
 * Many rules are governed at "subreddit" or group level. However, while anyone may start a subreddit and become an administrator, individual subreddits have formal mechanisms for rule creation and enforcement only by moderator users, as opposed to by the subreddit members as a whole.

Lessig, Lawrence. (1999). Code and other laws of cyberspace. New York, NY: Basic Books.\\
 * Lessig famously declared "code is law", speaking to quasi-governmental authority that systems administrators have over their digital domains and those who inhabit it.

\subsection{Implicit feudalism}

implicit feudalism [Schneider, Nathan and Calvin Liu 2019]

\section{Facebook's Oversight Board}

In 2019, Facebook began developing an Oversight Board\index{Facebook!Oversight Board}, which will have the final say in any controversial moderation decisions and will even be able to reverse the decisions of moderators. The board will be independent from Facebook (thus protecting Facebook from accusations of bias) and, in the interest of transparency, the board's decision making process, final decisions, and list of members will all be known to the public. [Clegg 2019, Facebook]\\

Margaret Levi Henry Farrell and Tim O'Reilly. 2018. "Mark Zuckerberg runs a nation-state, and he's the king." (April 2018)\\
 * Policies such as "Supreme Court", appeals process, independent oversight committee lack civic participation from users as stakeholders and decision-makers.

\section{Ostrom workshop}

Ostrom, E. 2000. "Collective Action and the Evolution of Social Norms." Journal of Economic Perspectives 14 (3): 137-58.\\
 * Rules created by a community itself are most effective.

Kou, Yubo and Bonnie Nardi. 2013. "Regulating Anti-Social Behavior on the Internet: The Example of League of Legends." (Feb. 12-15 2013), 616-622.\\
 * Crowdsourced approach to governance: League of Legends tribunals
 
Kou, Yubo et al. 2017. "Managing Disruptive Behavior Through Non-Hierarchical Governance: Crowdsourcing in League of Legends and Weibo." Proc. ACM Hum.-Comput. Interact. 1, CSCW, Article 62 (Dec. 2017), 17 pages.\\
 * Crowdsourced approach to governance: Weibo

Lampe, Cliff and Paul Resnick. 2004. "Slash(dot) and burn: distributed moderation in a large online conversation space." In Proceedings of the SIGCHI Conference on Human Factors in Computing Systems (CHI '04). ACM, 543-550.\\
 * Crowdsourced approach to governance: Slashdot moderators

Ullyot, Ted. 2009. "Results of the Inaugural Facebook Site Governance Vote." (Apr 2009).\\
 * Crowdsourced approach to governance: Facebook's short-lived policy voting system

\subsection{Participatory change}

participatory change [Frey, Seth et al. 2019]

\section{Modular politics}

modular politics [De Filippi, Primavera et al.]

\section{Decision resolution}

resolution [Im, Jane et al. 2018]

Butler, Brian et al. 2008. "Don't look now, but we've creatd a bureaucracy: the nature and roles of policies and rules in wikipedia." In Proceedings of the SIGCHI Conference on Human Factors in Computing Systems. ACM, 1101-1110.\\
 * Wikipedia's highly flexible governance has resulted in a system that is considered bureaucratic.

Muhlberger, Peter. 2005. "The Virtual Agora Project: A Research Design for Studying Democratic Deliberation." Journal of Public Deliberation 1, 1 (2005).\\
 * Virtual Agora project compares face-to-face deliberation with online deliberation, measuring its impact on decision quality and perceived legitimacy of choices.

Drapeau, Ryan et al. 2016. "MicroTalk: Using Argumentation to Improve Crowdsourcing Accuracy." In HCOMP.\\
 * MicroTalk system structures comments into discrete, persuasive arguments for one-round debate.

Schaekermann, Mike et al. 2018. "Resolvable vs. Irresolvable Disagreement: A Study on Worker Deliberation in Crowd Work." Proceedings of the ACM on Human-Computer Interaction 2, CSCW (2018), 1-19.\\
 * Expands on MicroTalk's structure with synchronous workflows for multi-term and contextual argumentation.

Chen, Quanze et al. 2018. "Cicero: Multi-Turn, Contextual Argumentation for Accurate Crowdsourcing." (2018).\\
 * Also expands on MicroTalk's structure with synchronous workflow for multi-term and contextual argumentation -- shows it leads to improved accuracy.

Fishkin, James et al. 2019. "Deliberative Democracy with the Online Deliberation Platform" (2019).\\
 * Stanford Online Deliberation Platform conducts group video sessions for civic deliberation using an automated facilitator.

\subsection{Digital juries}

digital juries [Fan, Jenny and Amy Zhang 2020]

Haewoon, Kwak et al. 2015. "Exploring cyberbullying and other toxic behavior in team competition online games." In Proceedings of the 33rd Annual ACM Conference on Human Factors in Computing Systems. ACM, 3739-3748.\\
 * In League of Legends tribunals -- governance system resembling digital juries in practice -- participants reported experience as helpful for learning norms and building community.

Kou, Yubo and Xinning Gui. 2017. "When Code Governs Community." In 50th Hawaii International Conference on Systems Sciences, HICSS 2017, Hilton Waikoloa Village, Hawaii, USA, January 4-7, 2017. 1-9.\\
 * However, Riot Games prioritized efficiency over legitimacy in design, leading to decisions such as a lack of transparency about cases or votes, random assignment of judges to cases, and no interaction between judges (despite user-initiated efforts to communicate).

\section{Reading questions}

\section{Solutions to reading questions}

\section{Bibliography}

Clegg, Nick. "Charting a Course for an Oversight Board for Content Decisions." Facebook. January 28, 2019.

De Filippi, Primavera et al. "Modular Politics: Toward a Governance Layer for Online Communities."

Fan, Jenny and Amy X. Zhang. "Digital Juries: A Civics-Oriented Approach to Platform Governance." CHI 2020, April 25-30, 2020.

Frey, Seth et al. "'This Place Does What It Was Built For': Designing Digital Institutions for Participatory Change." Proceedings of the ACM on Human-Computer Interaction. 32. November, 2019.

Im, Jane et al. "Deliberation and Resolution on Wikipedia: A Case Study of Requests for Comments." In Proceedings of the ACM on Human-Computer Interaction, Vol. 2, CSCW, Article 74. November, 2018.

Schneider, Nathan and Calvin Liu. "Admins, Mods, and Benevolent Dictators for Life: The Implicit Feudalism of Online Communities." 2019.

\end{document}
