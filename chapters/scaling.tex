\documentclass[class=book, crop=false]{standalone}

%% Sets page size and margins
\usepackage[a4paper,top=3cm,bottom=2cm,left=3cm,right=3cm,marginparwidth=1.75cm]{geometry}

%% Image paths
\usepackage{amsmath}
\usepackage{geometry}
\usepackage{graphicx}
\usepackage{svg}
\usepackage[font=small,labelfont=bf]{caption}
\graphicspath{{images/}}

%% Language and font encodings
\usepackage[english]{babel}
\usepackage[utf8x]{inputenc}
\usepackage[T1]{fontenc}

%% Sets epigraph style
\usepackage{epigraph}
\setlength\epigraphwidth{.8\textwidth}
\setlength\epigraphrule{0pt}

%% Sets line style
\linespread{1.3}

%% Key term command
\usepackage{marginnote}
\providecommand{\keyterm}[1]{\textbf{#1}\marginnote{\scriptsize \textbf{#1}}}

\begin{document}

\epigraph{\itshape "Winter is Coming!"}{---House Stark, \textit{Game of Thrones}}

\epigraph{\itshape "It's moot now. September 1993 will go down in net history as the September that never ended."}{---Dave Fischer, \textit{alt.folklore.computers}}

When a social system grows larger, newcomers may end up destroying the system’s norms\index{norm}. New members are typically more energetic than older members and many are also interested in shifting the focus of the community. [Jeffries 2005] Because newcomers aren’t as familiar with community norms, they are more likely to violate them.\footnote{Sometimes pwning noobs can be both fun \textit{and} helpful! (Just kidding)} [Kraut, Burke and Riedl 2012]\\

\section{The Eternal September}

In 1980, four computer scientists at Duke University: James Ellis, Steve Daniel, Tom Truscott and Steve Bellovin, created the early software for \keyterm{Usenet}\index{Usenet}. Usenet was one of the first social networks, and it was arguably \textit{the} first social network that really took off. Anybody with Unix could join Usenet by setting up a modem and acquiring a large amount of disk storage.

\begin{figure}[h]
    \centering
    \includesvg[width=0.5\textwidth]{usenet}
    \caption{Layout of Usenet servers and clients. The colored dots on the servers represent the newsgroups run by the server. Arrows between severs represent exchanges of info about newsgroups run by both and arrows between servers and clients represent that a client is in a newsgroup.}
\end{figure}

Usenet\index{Usenet} allowed users to create topics ranging from auto mechanics to religion and discuss them with other users in a discussion forum format not too unsimilar in function to Reddit today. These topic-organized groups were called \keyterm{newsgroups}\index{newsgroups}. Usenet quickly took off, with other universities and Bell Labs connecting to Usenet. [Emerson 1983]

However, while research and industrial institutions were able to connect to Usenet\index{Usenet}, the average person faced steep barriers in connecting to Usenet. While connecting to Usenet didn't require much hardware, you did need Internet access and the technical knowledge necessary to set up the Usenet software (not an easy feat in the 1980s). This kept the number of people who used Usenet relatively low and mostly consistent. However, every September when freshmen went to college and gained access to the Internet, they logged on to Usenet for the first time en masse. This influx of new users strained the culture\index{culture} of Usenet -- the freshmen didn't understand the norms\index{norm} underlying Usenet and frequently violated them. However, within a month or so, Usenet was able to absorb the influx of freshmen (or the freshmen left, intimidated by the Usenet culture) and its culture stabilized again.

This all changed in September of 1993, when AOL\index{AOL} gave millions of its customers access to Usenet\index{Usenet} for the first time. There were so many new users that when they violated Usenet's norms\index{norm}, there weren't enough veteran users to teach them proper Usenet etiquette. Usenet's culture\index{culture} was quickly destroyed in what regular Usenet user Dave Fischer called "the September that never ended." [Koebler 2015]

The destruction of Usenet's culture is the first online instance of what is now fittingly known as an \keyterm{Eternal September}\index{Eternal September} -- when so many new users join an online community that they end up destroying the community's culture\index{culture}.

\section{Strong moderation}

In order for an enlarging social system to stay afloat, it must have two things: strong \index{moderation} and increased underprovision of attention. [Kiene et al. 2016; Lin et al. 2017]\\

Strong moderation: Moderators must be vigilant
\begin{itemize}
    \item Moderating content or banning substantially decreases negative behavior in the short term on Twitch. [Seering et al. 2017]
    \item Reddit’s ban of /r/CoonTown and /r/fatpeoplehate for violating anti-harassment policy was successful. The accounts either left entirely or drastically reduced their hate speech. [Chandrasekharan et al. 2017]
    \item Community moderation -- likes, upvotes/downvotes, comments, reporting, muting
    \item AI bots that moderate content
    \item Shadow banning -- blocking or partially blocking a user from a social system so that it isn’t immediately apparent to the user that they are banned. Intent is to bore or frustrate the user so that they leave the system rather than simply creating a new account (as they might if they are banned normally).
\end{itemize}

\section{Information overload}

Increased underprovision of attention: Problem of information overload
\begin{itemize}
    \item As a subreddit\index{Reddit} gets larger, its users cluster their comments around a smaller and smaller proportion of posts. [Lin et al. 2017]
\end{itemize}

\section{Ranking and optimization}

\section{Reading questions}

\section{Solutions to reading questions}

\section{Bibliography}

Ahn, Yong-Yeol et al. "Analysis of topological characteristics of huge online social networking services." Proceedings of the 16th international conference on World Wide Web, pp. 835-844. May 8 - 12, 2007.

Chandrasekharan, Eshwar et al. "You Can't Stay Here: The Efficacy of Reddit's 2015 Ban Examined Through Hate Speech." Proceedings of the 2017 Association for Computing Machinery Human-Computer Interaction. November, 2017.

Danescu-Niculescu-Mizil, C. et al. "No country for old members: User lifestyle and linguistic change in online communities." Proceedings of the 22nd international conference on World Wide Web, 307-318. ACM. 2013.

Emerson, Sandra L. "Usenet / A Bulletin Board for Unix Users." BYTE. pp. 219-236. October, 1983.

Hu, Haibo and Wang, Xiaofan. "Evolution of a large online social network." Physics Letters A 373 (12-13): pp. 1105-1110. March 16, 2009.

Jeffries, Robin et al. "Systers: Contradictions in community." 2005.

Kiene, Charles et al. "Surviving an 'Eternal September' -- How an Online Community Managed a Surge of Newcomers." Proceedings of the 34th Annual ACM Conference on Human Factors in Computing Systems. May 28, 2016.

Koebler, Jason. "It's September, Forever." Vice. September 30, 2015.

Kraut, Robert et al. "Dealing with Newcomers." \textit{Evidenced-based social design: Mining the social sciences to build online communities}. Cambridge, MA: MIT press. 2012.

Lin, Zhiyu et al. "Explore, Exploit or Listen: Combining Human Feedback and Policy Model to Speed up Deep Reinforcement Learning in 3D Worlds." September 12, 2017.

Lin, Zhiyuan et al. "Better When It Was Smaller? Community Content and Behavior After Massive Growth." Association for the Advancement of Artificial Intelligence. 2017.

Seering, Joseph et al. "Shaping Pro and Anti-Social Behavior on Twitch Through Moderation and Example-Setting." Proceedings of the 2017 ACM Conference on Computer Supported Cooperative Work and Social Computing. 111-125. Feb 25 - Mar 1, 2017.

\end{document}
