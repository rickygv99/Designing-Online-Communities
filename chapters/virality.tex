\documentclass[class=book, crop=false]{standalone}

%% Image paths
\usepackage{graphicx}
\graphicspath{{images/}}

%% Language and font encodings
\usepackage[english]{babel}
\usepackage[utf8x]{inputenc}
\usepackage[T1]{fontenc}

%% Sets page size and margins
\usepackage[a4paper,top=3cm,bottom=2cm,left=3cm,right=3cm,marginparwidth=1.75cm]{geometry}

%% Sets epigraph style
\usepackage{epigraph}
\setlength\epigraphwidth{.8\textwidth}
\setlength\epigraphrule{0pt}

%% Sets line style
\linespread{1.3}

%% Key term command
\usepackage{marginnote}
\providecommand{\keyterm}[1]{\textbf{#1}\marginnote{\scriptsize \textbf{#1}}}

\begin{document}

\epigraph{\itshape "Virality isn't luck. It's not magic. And it's not random. There's a science behind why people talk and share. A recipe. A formula, even."}{---Jonah Berger, author of \textit{Contagious: Why Things Catch On}}

\epigraph{\itshape "Only one form of contagion travels faster than a virus. And that's fear."}{---Dan Brown, author of \textit{The Da Vinci Code}}

To figure out how to make something go viral, we must first figure out what \keyterm{virality} is. But before we formally define virality, we must note one thing. Virality is \textbf{community specific}. A phenomenon can go viral in one community but not another. For example, a meme might go viral on the Facebook group Subtle Asian Traits but not on Reddit.

Now, let's formally define what it means for a phenomenon to go viral. We know that viral phenomena spread quickly. Let's make that one of our criteria. That's not sufficient though; something can spread rapidly but only to a few people. When we think of something viral, we think of it dominating a community. Everybody's talking about it. Let's make that a requirement then, that viral phenomena saturate a community. Now, is this enough? Not quite yet. For example, mathematical knowledge spreads rapidly to incoming math students and saturates the mathematical community; however, we don't consider it viral phenomena. Most people in the community have known the mathematical knowledge for a long time. One final criteria, then, is that the community must initially be unsaturated by the viral phenomenon.

And so we arrive at the following three criteria for viral phenomena:
\begin{enumerate}
    \item The phenomenon must saturate a community.
    \item The phenomenon must spread rapidly.
    \item The community must be unsaturated by the phenomenon at first.
\end{enumerate}

For a phenomenon to start off relatively unknown and come to quickly saturate a community, people must have some incentive to spread it quickly. In this chapter, we will discuss how we can create phenomena that incentivize people to share them and how to do so ethically rather than through more nefarious ways such as spreading offensive or false content.

\section{Cultural innovation}

\keyterm{Cultural innovation}\index{cultural innovation}
\begin{itemize}
    \item Core (mainstream) and periphery (marginal communities) [Bynum et al. 1999]\index{community!core}\index{community!peripheral}
    \item Cultural innovation\index{cultural innovation} greatest among those who are bridges between core and periphery [Cattani and Ferriani 2008; Dahlander and Frederiksen 2012]
\end{itemize}

\section{Contagion}

In 2008, Jonah Berger and Katherine Milkman, both professors at Wharton, analyzed all \textit{New York Times} articles over a three month time period. To gauge the virality of the articles, they checked which emails made it on the most emailed list, which was updated every 15 minutes. In doing so, they discovered several ways that the emotional content of the articles affected whether the articles went viral:
\begin{itemize}
    \item Articles that evoked emotions were more likely to go viral.
    \item Articles that made people feel happy or inspired were more likely to go viral.
    \item Articles that made people feel sad were less likely to go viral.
    \item Articles that made people feel excited, anxious or angry were more likely to go viral. In particular, out of all the factors that influenced the virality of articles, evoking anger had the greatest effect on whether an article went viral.
    \item Articles that were useful or surprising were more likely to go viral.
\end{itemize}

Contagious: Why Things Catch On [Berger, Jonah 2016]

Three basic criteria: messenger, message, environment [Kaplan, Andreas M. and Michael Haenlein 2011]

\section{Metrics of virality}

K-factor [Fong, Richard 2014]

Social K-factor [Jagannathan, Anand 2017]

\section{Social proof}

\keyterm{Social proof}\index{social proof}
\begin{itemize}
    \item Looking up at a building experiment [Milgram, Bickman and Berkowitz 1969]
    \item The strength of social proof\index{social proof} varies across different cultures. Subjects in collectivist cultures conform to others’ social proof more often than those in individualist cultures. [Bond and Smith 1996]
    \item Copycat suicides
\end{itemize}

Herd behavior

Conformity

https://journals.sagepub.com/doi/full/10.1509/jmr.10.0353

\section{Hubs and influencers}

Alpha users / hubs

\begin{itemize}
    \item \keyterm{Social influence}\index{social influence} increases both inequality and unpredictability of success [Salganik, Dodds and Watts 2006]
    \item In experiment, best songs usually don’t do badly and worst songs usually don’t do well but any other result is possible [Salganik, Dodds and Watts 2006]
\end{itemize}

\section{Misinformation}

Advertisements, social media campaigns and online influencers often use the properties of virality to boost themselves or their sales. Political agents, conspiracy groups and nation-states use the same properties of virality, but to nefarious ends. One of the ways that they take advantage of virality is by spreading misinformation online.

In the past, it was more difficult to spread fake news. Without a platform, individual actors and groups had limited reach and couldn't easily influence many people. However, with the rise of the Internet, proliferators of fake news have found it easy to find an audience for their misinformation, and as a result, fake news frequently goes viral online [Soll 2016].

For example, take the flat earth movement. Before the Internet, the Flat Earth Society had only 3,500 members. Now, over 300,000 people visit their official website each day, Facebook pages promoting a flat Earth have received over 100,000 likes and millions of people have watched YouTube videos about the Earth being flat. [Weber 2018]

In fact, in an interviewer with CNN reporter Rob Picheta, YouTuber Mark Sargent, one of the leaders in the flat-Earth movement, claims that without YouTube, the flat Earth movement would not exist. "Flat Earth was a binge watch on YouTube," he said. Another flat-Earth leader, Robbie Davidson, said "Anything on social media is always going to be helpful if it goes viral, right? Well, flat Earth has gone viral." [Picheta 2019]

Fake news spreads more virally\index{virality} than true news not because bots are spreading it but because humans themselves are spreading it. Furthermore, fake news is significantly more likely to go viral than true news. Correlated with that is the fact that fake news is more surprising than true news, a likely factor correlated with whether the news goes viral or not. [Vosoughi, Roy and Aral 2018]

Sometimes politicians use the mantra of "fake news"\index{fake news} to falsely discredit media organizations and other politicians who release negative information about them [Mihailidis and Viotty 2017].

Older people are more likely to share fake news online than are younger people [Gallagher and Berger 2019].

Harvard professor Claire Wardle defines seven types of misinformation, which are listed below in order of "intent to deceive":
\begin{enumerate}
    \item \textbf{Satire or Parody}: Many satirical articles and websites (e.g. \textit{The Onion} and \textit{The Babylon Bee}) don't set out to mislead people. They even tend to explicitly state that their content is satirical. However, some people are still fooled by this satire, especially when it fits in with their pre-existing worldview (for some humorous examples of this, check out the Tumblr page \textit{Literally Unbelievable}).
    \item \textbf{False Connection}: Sometimes a headline, caption or image is deceptive, whether for shock value (e.g. clickbait) or for propaganda purposes.
    \item \textbf{Misleading Content}: Content is misleading when it is used to create a false or exaggerated depiction of its subject without outright saying anything false. Misleading content is typically generated for political or propaganda purposes.
    \item \textbf{False Context}: Sometimes people try to twist news stories by delivering content that is true (and thus can be verified by readers) but supplanted with false contextual information.
    \item \textbf{Imposter Content}: Whether for parody, profit or propaganda purposes, oftentimes fake news will falsely attribute content to well known figures.
    \item \textbf{Manipulated Content}: While this type of fake news still contains some genuine content, it is warped and twisted with false information.
    \item \textbf{Fabricated Content}: This fake news simply doesn't contain any genuine information and is used for a variety of reasons: parody, provocation, profit, political benefit or propaganda purposes.
\end{enumerate}

Detecting fake news

Clickbait

\section{Reading questions}

\section{Solutions to reading questions}

\section{Bibliography}

Berger, Jonah. \textit{Contagious: Why Things Catch On}. Simon \& Schuster. 2016.

Berger, Jonah and Milkman, Katherine L. "What Makes Online Content Viral?" Journal of Marketing Research. 49(2):192-205. April, 2012.

Bond, R. and Smith, P. B. "Culture and conformity: A meta-analysis of studies using Asch's line judgment task." Psychological Bulletin. 119(1):111-137. 1996.

Cattani, G. and Ferriani S. "A Core/Periphery Perspective on Individual Creative Performance: Social Networks and Cinematic Achievements in the Hollywood Film Industry." Organization Science. 19(6):824-844. November 21, 2006.

Dawkins, Richard. \textit{The Selfish Gene}. Oxford University Press. 1976.

Feroz, Khan G. and Vong, S. "Virality over YouTube: an empirical analysis." Internet Research 24 (5): pp. 629-647. September 30, 2014.

Fong, Richard. "The K-Factor: The Secret Factor Behind Your Company's Growth." BlissDrive. March 17, 2014.

Gallagher, Brian and Berger, Kevin. "Why Misinformation Is About Who You Trust, Not What You Think." Nautilus. February 14, 2019.

Goel, Sharad et al. "The Structural Virality of Online Diffisuion." Management Science 62 (1): pp. i-vii, 1-301. July 22, 2015.

Ho, Jason Y. C. and Dempsey, Melanie. "Viral marketing: Motivations to forward online content." Journal of Business Research 63 (9-10): pp. 1000-1006. September - October, 2010.

Jagannathan, Anand. "The Social K-factor: Tracking Viral Growth in a Social World." Medium. April 27, 2017.

Kaplan, Andreas M. and Michael Haenlein. "Two hearts in three-quarter time: How to waltz the Social Media/viral marketing dance." Business Horizons. 54(3), pp. 253-263. 2011.

Killian, Lewis M. et al. "Collective behavior." Encyclopedia Britannica. January 29, 2018.

Lazer, David M. J. et al. "The science of fake news." Science 359 (6380): pp. 1094-1096. March 9, 2018.

Marchi, Regina. "With Facebook, Blogs, and Fake News, Teens Reject Journalistic 'Objectivity'." Journal of Communication Inquiry 36 (3): pp. 246-262. July 1, 2012.

McCarthy, A. J. "'Numa Numa,' the Original Viral Video, Turns 10." Slate. December 5, 2014.

Mihailidis, Paul and Vitty, Samantha. "Spreadable Spectacle in Digital Culture: Civic Expression, Fake News, and the Role of Media Literacies in 'Post-Face' Society." American Behavioral Scientist. 61(4):441-454. March 27, 2017.

Milgram, Stanley et al. "Note on the drawing power of crowds of different size." Journal of Personality and Social Psychology. 13(2):79-82. 1969.

Mills, Adam J. "Virality in social media: the SPIN Framework." Journal of Public Affairs 12 (2). March 21, 2012.

Picheta, Rob. "The flat-Earth conspiracy is spreading around the globe. Does it hide a darker core?" CNN. November 18, 2019.

Salganik, Matthew J. et al. "Experimental Study of Inequality and Unpredictability in an Artificial Cultural Market." Science. 311(5762):854-856. February 10, 2006.

Soll, Jacob. "The Long and Brutal History of Fake News." POLITICO Magazine. December 18, 2016.

Vosoughi, Soroush et al. "The spread of true and false news online." Science. 359(6380):1146-1151. March 9, 2018.

Wardle, Claire. "Fake news. It's complicated." First Draft News. February 16, 2017.

Weber, Matt J. "How the Internet Made Us Believe in a Flat Earth." Medium. December 12, 2018.

\end{document}
